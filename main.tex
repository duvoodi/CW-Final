\documentclass[titlepage]{article}

\usepackage{fancyhdr}
\pagestyle{fancy}

\usepackage{xcolor}
\definecolor{Gray}{rgb}{0.5,0.5,0.5} % خاکستری

\usepackage[pdfborder={0 0 0}, bookmarks=true]{hyperref}
% آپشن اول بردر ها لینک را حذف میکند و باید دستی لینک ها را اندر لاین کرد

\usepackage{graphicx}
\graphicspath{ {./img/} }

\usepackage[most]{tcolorbox}

\usepackage{changepage}

\usepackage{listings}
\lstset{
  basicstyle=\ttfamily\small,           % فونت اصلی برای کد
  numbers=left,                         % شماره خطوط در سمت چپ
  numberstyle=\tiny\color{Gray},        % استایل شماره خطوط
  breaklines=true,                      % شکستن خطوط بلند
  breakatwhitespace=true,               % شکستن خطوط در محل فاصله‌ها
  showstringspaces=false,               % حذف فاصله‌های خاص در رشته‌ها
  tabsize=2,                            % اندازه تب
  captionpos=b,                         % موقعیت عنوان (بالا/پایین)
}

\usepackage{comment}

\begin{document}

\tcbset{
  inline/.style={
    colframe=gray,                     % رنگ قاب
    colback=gray!10,                   % رنگ پس‌زمینه
    fonttitle=\bfseries,               % استایل عنوان
    boxrule=0.4pt,                     % ضخامت قاب
    left=-1pt, right=-1pt,             % فاصله متن از چپ و راست
    top=0pt, bottom=0pt,               % فاصله متن از بالا و پایین
    height=10.9pt,                     % ارتفاع ثابت باکس
    valign=center,                     % تراز عمودی محتوا در باکس
    baseline=2.9pt,                    % تنظیم خط پایه باکس
    before=\vspace{0pt},               % فاصله قبل از باکس
    after=\vspace{0pt},                % فاصله بعد از باکس
    enhanced,                          % فعال‌سازی تنظیمات پیشرفته
    breakable,                         % اجازه شکستن خطوط
    fontupper=\ttfamily\small,         % تنظیم فونت محتوای داخل باکس
  },
}
\tcbset{
  codebox/.style={
    colframe=gray,                     % رنگ قاب
    colback=gray!10,                   % رنگ پس‌زمینه
    boxrule=0.4pt,                     % ضخامت قاب
    left=1pt, right=1pt,               % فاصله از چپ و راست
    top=-4pt, bottom=-4pt,             % فاصله از بالا و پایین
    valign=center,                     % تراز عمودی محتوا در باکس
    enhanced,                          % فعال‌سازی تنظیمات پیشرفته
    breakable,                         % اجازه شکستن خطوط
  },
}
\tcbset{
  picframe/.style={
    colframe=gray,                     % رنگ قاب
    colback=gray!10,                   % رنگ پس‌زمینه
    boxrule=0.4pt,                     % ضخامت قاب
    left=1pt, right=1pt,               % فاصله از چپ و راست
    top=1pt, bottom=1pt,               % فاصله از بالا و پایین
    valign=center,                     % تراز عمودی محتوا در باکس
    enhanced,                          % فعال‌سازی تنظیمات پیشرفته
    breakable,                         % اجازه شکستن خطوط
  },
}

\title{\textbf{Computer Workshop\\Final Assignment}}
\author{Dr. MalekiMajd}
\date{Amir Mohammad Davoodi\\Winter of 4031}
\maketitle

\fancyhead{}
\fancyhead[L]{Final Assignment}
\fancyhead[R]{Computer Workshop Course}
\fancyfoot{}
\fancyfoot[C]{Page \thepage}

\tableofcontents 

\pagebreak

\section{Git and GitHub}

\end{document}