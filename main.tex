\documentclass[titlepage]{article}

\usepackage{fancyhdr}
\pagestyle{fancy}

\usepackage{xcolor}
\definecolor{Gray}{rgb}{0.5,0.5,0.5} % خاکستری

\usepackage[pdfborder={0 0 0}, bookmarks=true]{hyperref}
% آپشن اول بردر ها لینک را حذف میکند و باید دستی لینک ها را اندر لاین کرد

\usepackage{graphicx}
\graphicspath{ {./img/} }

\usepackage[most]{tcolorbox}

\usepackage{changepage}

\usepackage{listings}
\lstset{
  basicstyle=\ttfamily\small,           % فونت اصلی برای کد
  numbers=left,                         % شماره خطوط در سمت چپ
  numberstyle=\tiny\color{Gray},        % استایل شماره خطوط
  breaklines=true,                      % شکستن خطوط بلند
  breakatwhitespace=true,               % شکستن خطوط در محل فاصله‌ها
  showstringspaces=false,               % حذف فاصله‌های خاص در رشته‌ها
  tabsize=2,                            % اندازه تب
  captionpos=b,                         % موقعیت عنوان (بالا/پایین)
}

\usepackage{comment}

\begin{document}

\tcbset{
  inline/.style={
    colframe=gray,                     % رنگ قاب
    colback=gray!10,                   % رنگ پس‌زمینه
    fonttitle=\bfseries,               % استایل عنوان
    boxrule=0.4pt,                     % ضخامت قاب
    left=-1pt, right=-1pt,             % فاصله متن از چپ و راست
    top=0pt, bottom=0pt,               % فاصله متن از بالا و پایین
    height=10.9pt,                     % ارتفاع ثابت باکس
    valign=center,                     % تراز عمودی محتوا در باکس
    baseline=2.9pt,                    % تنظیم خط پایه باکس
    before=\vspace{0pt},               % فاصله قبل از باکس
    after=\vspace{0pt},                % فاصله بعد از باکس
    enhanced,                          % فعال‌سازی تنظیمات پیشرفته
    breakable,                         % اجازه شکستن خطوط
    fontupper=\ttfamily\small,         % تنظیم فونت محتوای داخل باکس
  },
}
\tcbset{
  codebox/.style={
    colframe=gray,                     % رنگ قاب
    colback=gray!10,                   % رنگ پس‌زمینه
    boxrule=0.4pt,                     % ضخامت قاب
    left=1pt, right=1pt,               % فاصله از چپ و راست
    top=-4pt, bottom=-4pt,             % فاصله از بالا و پایین
    valign=center,                     % تراز عمودی محتوا در باکس
    enhanced,                          % فعال‌سازی تنظیمات پیشرفته
    breakable,                         % اجازه شکستن خطوط
  },
}
\tcbset{
  picframe/.style={
    colframe=gray,                     % رنگ قاب
    colback=gray!10,                   % رنگ پس‌زمینه
    boxrule=0.4pt,                     % ضخامت قاب
    left=1pt, right=1pt,               % فاصله از چپ و راست
    top=1pt, bottom=1pt,               % فاصله از بالا و پایین
    valign=center,                     % تراز عمودی محتوا در باکس
    enhanced,                          % فعال‌سازی تنظیمات پیشرفته
    breakable,                         % اجازه شکستن خطوط
  },
}

\title{\textbf{Computer Workshop\\Final Assignment}}
\author{Dr. MalekiMajd}
\date{Amir Mohammad Davoodi\\Winter of 4031}
\maketitle

\fancyhead{}
\fancyhead[L]{Final Assignment}
\fancyhead[R]{Computer Workshop Course}
\fancyfoot{}
\fancyfoot[C]{Page \thepage}

\tableofcontents 

\pagebreak

\section{Git and GitHub}

\subsection{Repository Initialization and Commits}
\begin{enumerate}
\item \textbf{Create a GitHub Repository}  
\\First, I created a new repository on GitHub. This repository would contain my \LaTeX \hspace{0pt} source files and the configuration for the workflow.
\item \textbf{Clone the Repository Locally} 
\\To work on the repository locally, I cloned it to my computer using the following command:  
\begin{adjustwidth}{1cm}{1cm}
\begin{tcolorbox}[codebox]
\begin{lstlisting}[numbers=none]
  git clone <repo_link>
\end{lstlisting}
\end{tcolorbox}
\end{adjustwidth}
\item \textbf{Local Changes and Git Operations}
\\After that, I made my changes locally and in each step, I added and committed changes, and pushed them to my GitHub repository, with the following commands:
\begin{adjustwidth}{1cm}{1cm}
\begin{tcolorbox}[codebox]
\begin{lstlisting}[numbers=none]
  git commit -am "commit message"
\end{lstlisting}
\end{tcolorbox}
\end{adjustwidth}
\begin{adjustwidth}{1cm}{1cm}
\begin{tcolorbox}[codebox]
\begin{lstlisting}[numbers=none]  
  git push --all
\end{lstlisting}
\end{tcolorbox}
\end{adjustwidth}
\end{enumerate}

\subsection{GitHub Actions for \LaTeX \hspace{0pt} Compilation}
In my local repository, I created a GitHub Actions workflow file named \tcbox[inline]{main.yml} in the \tcbox[inline]{.github/workflows/} directory. This file automates the process of compiling the \LaTeX \hspace{0pt} document into a PDF and releasing it whenever a new tag is pushed. 
\vspace{5pt}
\\I set up the workflow file as follows:
\begin{tcolorbox}[codebox]
\begin{lstlisting}
name: Release Compiled PDF 
on:
  push:
    tags:
      - '*.*.*'

jobs:
  build_latex:
    permissions: write-all
    runs-on: ubuntu-latest
    steps:
      - name: Set up Git repository
        uses: actions/checkout@v2
      - name: Compile LaTeX document
        uses: xu-cheng/latex-action@v2
        with:
          root_file: main.tex

      - name: Create Release
        id: create_release
        uses: actions/create-release@v1
        env:
          GITHUB_TOKEN: ${{ secrets.GITHUB_TOKEN }}
        with:
          tag_name: ${{ github.ref }}
          release_name: Release ${{ github.ref }}
          draft: false
          prerelease: false

      - name: Upload Release Asset
        id: upload-release-asset 
        uses: actions/upload-release-asset@v1
        env:
          GITHUB_TOKEN: ${{ secrets.GITHUB_TOKEN }}
        with:
          upload_url: ${{ steps.create_release.outputs.upload_url }} 
          asset_path: ./main.pdf
          asset_name: main.pdf
          asset_content_type: pdf
\end{lstlisting}  
\end{tcolorbox}
\vspace{5pt}
\noindent Here's a breakdown of what each part does:
\begin{itemize}
\item \textbf{Workflow Trigger (on)}
\\ The workflow is triggered when a tag with a version number (in the format of \tcbox[inline]{*.*.*}, such as \tcbox[inline]{1.0.0}) is pushed to the repository.
\item \textbf{Job (build\_latex)}
\\ This job runs on an Ubuntu-based virtual machine (\tcbox[inline]{runs-on: ubuntu-latest}).
\item \textbf{Steps:}
\begin{itemize}
\item Set up Git repository: Uses the \tcbox[inline]{actions/checkout@v2} action to clone the repository so that the workflow can access the \LaTeX \hspace{0pt} files.
\item Compile \LaTeX \hspace{0pt} file: Uses the \tcbox[inline]{xu-cheng/latex-action@v2} action to compile the \LaTeX \hspace{0pt} document \tcbox[inline]{main.tex} into a PDF. This step produces a \tcbox[inline]{main.pdf} file.
\item Create Release: Uses the \tcbox[inline]{actions/create-release@v1} action to create a GitHub release with the tag name based on the version pushed (e.g., \tcbox[inline]{1.0.0}). It sets the release as non-draft and non-prerelease, and attaches a release name (e.g., Release \tcbox[inline]{1.0.0}).
\item Upload Release Asset: Uses the \tcbox[inline]{actions/upload-release-asset@v1} action to upload the compiled PDF \tcbox[inline]{main.pdf} as an asset to the release created in the previous step. This allows users to download the PDF from the release page.
\end{itemize}
\end{itemize}

\end{document}